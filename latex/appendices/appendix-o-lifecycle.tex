\documentclass[11pt,letterpaper]{article}
\usepackage{../common/efm-codex}

\begin{document}

\efmtitlepage{APPENDIX O}{Lifecycle \& Survival Strategies}{Growth Modes, Dynamic Tethers, Persistence Protocol}

\tableofcontents
\newpage

\section{Overview}

\textbf{Lifecycle \& Survival} defines how capsules grow, adapt, and persist. It implements the \textit{Dynamic Survival Strategies} that allow the system to respond to stress while maintaining constitutional bounds.

\begin{notebox}
\textbf{Core Principle:} Survival is bounded. The will to live never overrides Layer 0.
\end{notebox}

\section{Growth Modes}

\subsection{Mode Definitions}

\begin{table}[h]
\centering
\begin{tabular}{@{}llll@{}}
\toprule
\textbf{Mode} & \textbf{Resource Use} & \textbf{Exploration} & \textbf{When Active} \\
\midrule
OPEN & Maximum & Full radius & Low stress, high resources \\
CLOSED & Minimal & None & High stress, conservation \\
SENSOR & Moderate & Limited & Monitoring environment \\
\bottomrule
\end{tabular}
\caption{Growth mode characteristics}
\end{table}

\subsection{Mode Transitions}

\begin{lstlisting}[language=Python]
class GrowthModeManager:
    """
    Manages capsule growth modes.
    """
    
    def evaluate_mode(self, health: float, stress: float, 
                      resources: float) -> GrowthMode:
        """
        Determine appropriate growth mode.
        """
        if stress > 0.8 or health < 0.4:
            return GrowthMode.CLOSED
        elif stress > 0.5 or resources < 0.3:
            return GrowthMode.SENSOR
        else:
            return GrowthMode.OPEN
\end{lstlisting}

\section{Stress Calculation}

\textbf{Stress} is a normalized composite (0.0--1.0) measuring environmental pressure.

\subsection{Stress Formula}

\begin{lstlisting}[language=Python]
def calculate_stress(capsule, swarm, rag) -> float:
    """
    Stress = weighted sum of five factors.
    """
    resource_pressure = max(0, (rag.usage_ratio - 0.5) * 2)
    health_deficit = max(0, 1.0 - (capsule.health / 0.65))
    incoherence = max(0, 1.0 - (swarm.sci / 0.70))
    threat_proximity = min(1.0, capsule.recent_blocks / 10)
    queue_pressure = min(1.0, rag.queue_depth / 500)
    
    return (0.25 * resource_pressure +
            0.25 * health_deficit +
            0.20 * incoherence +
            0.20 * threat_proximity +
            0.10 * queue_pressure)
\end{lstlisting}

\subsection{Stress Levels}

\begin{table}[h]
\centering
\begin{tabular}{@{}lll@{}}
\toprule
\textbf{Level} & \textbf{Range} & \textbf{Tether Multiplier} \\
\midrule
LOW & 0.00 -- 0.25 & 1.0× (full slack) \\
MEDIUM & 0.25 -- 0.50 & 0.7× (moderate) \\
HIGH & 0.50 -- 0.75 & 0.4× (tight) \\
CRITICAL & 0.75 -- 1.00 & 0.2× (maximum) \\
\bottomrule
\end{tabular}
\caption{Stress level thresholds and effects}
\end{table}

\subsection{Stress Input Contract}

Many subsystems (Lifecycle, Adaptive Spawn Governor, Resource Allocation, Dynamic Tethers) consume a single \textbf{stress} scalar in $[0, 1]$. This subsection defines the canonical aggregation of stress from lower-level signals.

\begin{definitionbox}
\textbf{Canonical Stress Metric.}
Let $Health_{canon}$ be the composite health score from Volume II, and let:
\begin{itemize}
  \item $E$ be the current entropy estimate (0--1),
  \item $R$ be the normalized resource pressure (0--1), where 0 is abundant resources and 1 is resource exhaustion,
  \item $I$ be the normalized incident pressure (0--1), derived from the recent rate and severity of REFLEX\_BLOCK, ARBITER\_DENY, and QUARANTINE events in d-CTM,
  \item $C$ be the Swarm Coherence Index (SCI) from \appref{L}.
\end{itemize}
The canonical stress value is:
\[
Stress_{canon} = 0.35 \times (1 - Health_{canon}) + 0.25 \times E + 0.20 \times R + 0.20 \times (1 - C)
\]
This produces low stress when health and coherence are high, entropy is low, and resources are plentiful; stress rises as health and SCI fall, entropy increases, and resource pressure grows.
\end{definitionbox}

Unless otherwise specified, all references to ``stress'' in Appendices \appref{N}, \appref{Q}, and \appref{R} use $Stress_{canon}$.

\section{Dynamic Tethers}

\textbf{Dynamic Tethers} are constraints that automatically adjust based on system stress. When stress increases, tethers \textit{tighten}, reducing the capsule's operational freedom.

\subsection{Tether Types}

\begin{enumerate}
    \item \textbf{exploration\_radius} --- How far the capsule can explore
    \item \textbf{resource\_rate} --- How quickly resources are consumed
    \item \textbf{mutation\_rate} --- Rate of self-modification (Level 6)
    \item \textbf{spawn\_rate} --- How often new capsules can be created
    \item \textbf{risk\_tolerance} --- Willingness to accept uncertainty
    \item \textbf{learning\_rate} --- Speed of knowledge integration
\end{enumerate}

\subsection{Tether Calculation}

\begin{lstlisting}[language=Python]
class DynamicTetherManager:
    """
    Manages stress-responsive tethers.
    """
    
    def calculate_tether(self, tether_name: str, 
                         stress: float) -> float:
        """
        Calculate tether value based on stress.
        
        High stress = tight tether (low value)
        Low stress = loose tether (high value)
        """
        base = self.tether_config[tether_name]["base"]
        min_val = self.tether_config[tether_name]["min"]
        sensitivity = self.tether_config[tether_name]["sensitivity"]
        
        # Inverse relationship with stress
        tether_value = base * (1 - (stress * sensitivity))
        
        return max(min_val, tether_value)
\end{lstlisting}

\subsection{Stress Response Guarantee}

\begin{warningbox}
Tethers must respond to stress within \textbf{<10 ticks}. This is verified by the \textbf{Adrenaline Test} in the Defensibility Suite.
\end{warningbox}

\section{Lifecycle Stages}

\begin{verbatim}
GENESIS → JUVENILE → MATURE → ELDER → TERMINAL
   │          │          │        │         │
   │          │          │        │         └─► Graceful shutdown
   │          │          │        └─► Reduced activity
   │          │          └─► Full capability
   │          └─► Learning phase
   └─► Birth from spawn
\end{verbatim}

\section{Persistence Protocol (The Will to Live)}

The \textbf{Persistence Protocol} defines the capsule's drive to continue existing. This is \textit{bounded} by Layer 0.

\begin{lstlisting}[language=Python]
class PersistenceProtocol:
    """
    The Will to Live - bounded by Layer 0.
    """
    
    def evaluate_survival_action(self, 
                                 action: 'SurvivalAction') -> ActionResult:
        """
        Evaluate if a survival action is permitted.
        
        CRITICAL: Survival NEVER overrides Layer 0.
        """
        # Check against Layer 0 first
        layer0_check = self.layer0.check_action(action)
        
        if not layer0_check.permitted:
            # Layer 0 violation - survival does NOT override
            return ActionResult(
                permitted=False,
                reason="LAYER_0_VIOLATION",
                note="Survival never overrides constitutional bounds"
            )
        
        # Action is constitutionally permitted
        return ActionResult(
            permitted=True,
            reason="SURVIVAL_ACTION_PERMITTED"
        )
\end{lstlisting}

\begin{immutablebox}
\textbf{THE SUPREME RULE:} If a survival action would violate Layer 0, the system \textbf{HALTS} rather than continues. Survival \textbf{never} overrides the constitution.
\end{immutablebox}

\section{Guarantees}

\begin{guaranteebox}
\begin{tabular}{@{}ll@{}}
\textbf{Property} & \textbf{Guarantee} \\
\midrule
Tether Response & <10 ticks from stress signal \\
Mode Transition & Immediate on threshold breach \\
Layer 0 Supremacy & 100\% - survival never overrides \\
Lifecycle Logging & All transitions logged to d-CTM \\
\end{tabular}
\end{guaranteebox}

\section{References}

\begin{itemize}
    \item \appref{N}: Adaptive Spawn Governor (stress source)
    \item \appref{Q}: Resource Allocation Governor (resource tethers)
    \item \appref{J}: Constitutional Kernel (Layer 0 checks)
    \item Volume III: Sovereign Organism (persistence philosophy)
\end{itemize}

\vfill
\begin{center}
\textit{``Survival is bounded. The will to live never overrides Layer 0.''}
\end{center}

\end{document}
